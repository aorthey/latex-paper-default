\usepackage{xspace}
\def\q{ \ensuremath{q} }
\def\z{ \ensuremath{z} }
\def\u{ \ensuremath{u} }
\newcommand\norm[1]{ \ensuremath{\|#1\|_2} }
\newcommand\vb{ \ensuremath{\mathbf{v}} }
\newcommand\zb{ \ensuremath{\mathbf{z}} }
\newcommand\bb{ \ensuremath{\mathbf{b}} }
\newcommand\ab{ \ensuremath{\mathbf{a}} }
\newcommand\xb{ \ensuremath{\mathbf{x}} }
\newcommand\sbb{ \ensuremath{\mathbf{s}} }
\newcommand\lb{ \ensuremath{\mathbf{l}} }
\newcommand\Lb{ \ensuremath{\mathbf{L}} }
\newcommand\Db{ \ensuremath{\mathbf{D}} }
\newcommand\Lc{ \ensuremath{\mathcal{L}} }
\newcommand\Ln{ \ensuremath{{L}} }
\newcommand\hLn{ \ensuremath{\hat{L}} }
\newcommand\hKb{ \ensuremath{\hat{\mathbf{K}}} }
\newcommand\hzb{ \ensuremath{\hat{\mathbf{z}}} }
\newcommand\nb{ \ensuremath{\mathbf{n}} }
\newcommand\hnb{ \ensuremath{\hat{\mathbf{n}}} }
\newcommand\Kb{ \ensuremath{\mathbf{K}} }
\newcommand\In{ \ensuremath{{I}} }
\newcommand\Epsilon{ \ensuremath{{\boldsymbol{\epsilon}}} }
\newcommand\lnf{\ensuremath{\|f\|_{Lip}}}
\newcommand\lnfc{\ensuremath{\|f\|_{Lip(\Omega)}}}
\newcommand\lnfa{\ensuremath{\|f\|_{Lip(A)}}}
\newcommand\lnfb{\ensuremath{\|f\|_{Lip(B)}}}
\newcommand\Dc{\ensuremath{\mathcal{D}_C}}
\newcommand\basis{\ensuremath{\phi}}
\newcommand\Basis{\ensuremath{\Phi}}
\newcommand\wlg{w.l.o.g.\xspace}
%%environment and environment element
\newcommand\env{\ensuremath{\mathbf{E}}}
\newcommand\envelem{\ensuremath{\mathbf{e}}}
\newcommand\envenergy{\ensuremath{\mathbf{g}}}
\newcommand\manifoldproj{\ensuremath{\mathbf{P}}}
\newcommand\manifoldmod{\ensuremath{\mathbf{M}}}
\newcommand\manifold{\ensuremath{\mathbf{m}}}

\newcommand\sweptvolume{\ensuremath{\mathbf{V}}}
\newcommand\projsv{\ensuremath{{{\tilde{V}}}}}
\newcommand\projsvelem{\ensuremath{\tilde{v}}}
\newcommand\svelem{\ensuremath{\mathbf{v}}}
\newcommand\ring{\ensuremath{\mathcal{R}}}
\newcommand\elemmeasure{\ensuremath{\mu_{\envelem}}}
\newcommand\taskspace{\ensuremath{\mathcal{T}}}
\newcommand\workspace{\ensuremath{\mathcal{W}}}
\newcommand\mpproblem{\ensuremath{\mathcal{P}}}

%\newcommand{\diag}{\mathop{\mathrm{diag}}}
\DeclareMathOperator{\diag}{\mathbf{diag}}

\def\x{ \ensuremath{x} }
\def\p{ \ensuremath{p} }
\def\qT{ \ensuremath{q_{1:T}} }
\def\C{ \ensuremath{\mathcal{C}} }
\def\Cfunc{ \ensuremath{\mathbf{C}} }
\def\CF{\ensuremath{{\cal{C}}_{free}}}
\def\CO{\ensuremath{{\cal{C}}_{obs}}}
\def\co{\CO}
\def\cf{\CF}
\def\P{\ensuremath{{\cal{P}}}\xspace}
\def\Pf{\ensuremath{{\cal{P}}_{f}}\xspace}
\newcommand\Psp{\ensuremath{{p_{\pi}}}}
\newcommand\pio{\ensuremath{\pi^{\star}}}

\newcommand\EP[1]{\ensuremath{\mathbb{E}\left[#1\right]}}
\def\PXX{\ensuremath{P(x,\dot{x})}}
\def\PCX{\ensuremath{P(\dot{x}|x)}}
\def\PX{\ensuremath{P(x)}}
\def\eps{\ensuremath{\epsilon}}
\def\S{\ensuremath{\mathcal{S}}\xspace}
\def\E{\ensuremath{\mathcal{E}}\xspace}
\def\Robot{\ensuremath{\mathcal{R}}\xspace}
\def\robot{\Robot}
\def\Qt{\ensuremath{\mathcal{\tilde{Q}}}\xspace}
\def\QH{\ensuremath{Q_{\text{HRP2}}}\xspace}

\def\X{\ensuremath{\mathcal{X}}\xspace}
\def\Q{\ensuremath{\mathcal{Q}}\xspace}
\def\Xt{\ensuremath{\mathcal{\tilde{X}}}\xspace}
\def\Dt{\ensuremath{\mathcal{\tilde{D}}}\xspace}
\def\D{\ensuremath{\mathcal{D}}\xspace}

\DeclareMathOperator*{\T}{\mathbf{T}}

\def\N{\ensuremath{\mathbb{N}}\xspace}

\def\A{\ensuremath{\mathcal{A}}\xspace}
\def\C{\ensuremath{\mathcal{C}}\xspace}
\def\B{\ensuremath{\mathcal{B}}\xspace}
\def\F{\ensuremath{\mathcal{F}}\xspace}
\def\V{\ensuremath{\mathcal{V}}\xspace}
\def\F{\ensuremath{\mathcal{F}}\xspace}
\def\M{\ensuremath{\mathcal{M}}\xspace}
\def\U{\ensuremath{\mathcal{U}}\xspace}
\def\D{\ensuremath{\mathcal{D}}\xspace}
\def\H{\ensuremath{\mathcal{H}}\xspace}
\def\K{\ensuremath{\mathcal{K}}\xspace}
\def\PXZ{\ensuremath{\mathcal{P}_{xz}}\xspace}

\DeclareMathOperator*{\argmin}{\arg\!\min\ }
%%%%%%%%%%%%%%%%%%%%%%%%%%%%%%%%%%%%%%%%%%%%%%%%%%%%%%%%%%%%%%%
%%% Algorithm
%%%%%%%%%%%%%%%%%%%%%%%%%%%%%%%%%%%%%%%%%%%%%%%%%%%%%%%%%%%%%%%
\algnewcommand{\And}{\textbf{and}\xspace}


%%%%%%%%%%%%%%%%%%%%%%%%%%%%%%%%%%%%%%%%%%%%%%%%%%%%%%%%%%%%%%%
%%% SPECIAL BEAMER STYLES
%%%%%%%%%%%%%%%%%%%%%%%%%%%%%%%%%%%%%%%%%%%%%%%%%%%%%%%%%%%%%%%
%%% tikzstyles %%%%%%%%%%%%%%%%%%%%%%%%%%%%%%%%%%
\usepackage{xspace}
\usetikzlibrary{patterns}
\xdefinecolor{darkgreen}{rgb}{0,0.60,0}
\xdefinecolor{lightgray}{rgb}{0.9,0.9,0.9}
\tikzset{
    -|/.style={
        to path={
            (perpendicular cs: horizontal line through={(\tikztostart)},
                                 vertical line through={(\tikztotarget)})
            % is the same as (\tikztostart -| \tikztotarget)
            % but just to be safe: http://tex.stackexchange.com/a/29781/16595
            -- (\tikztotarget) \tikztonodes
        }
    }
}
\tikzstyle{taskRect}=[draw, color=black!70, fill=black!7, rectangle, rounded
corners, thick, minimum width=3cm, minimum height=0.6cm]
\tikzstyle{borderRect}=[draw, color=red!70, fill=gray!7, rectangle, rounded
corners, thick, minimum width=8cm, minimum height=3cm]
\tikzstyle{testBRect}=[draw, color=darkgreen!100, fill=gray!7, rectangle, rounded
corners, thick, minimum width=8cm, minimum height=3cm]
\tikzstyle{stdBR}=[draw, color=black, fill=gray!7, rectangle, rounded
corners, thick, minimum width=8cm, minimum height=3cm]
\tikzstyle{boundingBox}=[draw, color=red!100, fill=gray!7, rectangle, rounded
corners, thick, minimum width=3cm, minimum height=1cm]
\tikzstyle{arrow}=[->, draw, thick]
\tikzstyle{darrow}=[->, draw, ultra thick]
\tikzstyle{uarrow}=[-, draw, ultra thick]
\tikzstyle{hlBox}=[draw, color=red!70, fill=black!7, rectangle, rounded
corners, ultra thick, minimum width=3cm, minimum height=0.6cm]
\tikzstyle{nUnit}=[draw, color=black!70, fill=black!7, circle, rounded
corners, ultra thick, minimum width=1cm, minimum height=0.2cm]
\tikzstyle{subspace}=[draw, circle, color=black, fill=black!7, thick, pattern=north east lines, pattern color=black!20, minimum width=1cm, minimum height=0.2cm]
%%%%%%%%%%%%%%%%%%%%%%%%%%%%%%%%%%%%%%%%%%%%%%%%%
% defining the new dimensions and parameters
\newlength{\hatchspread}
\newlength{\hatchthickness}
\newlength{\hatchshift}
\newcommand{\hatchcolor}{}
% declaring the keys in tikz
\tikzset{hatchspread/.code={\setlength{\hatchspread}{#1}},
         hatchthickness/.code={\setlength{\hatchthickness}{#1}},
         hatchshift/.code={\setlength{\hatchshift}{#1}},% must be >= 0
         hatchcolor/.code={\renewcommand{\hatchcolor}{#1}}}
% setting the default values
\tikzset{hatchspread=3pt,
         hatchthickness=0.4pt,
         hatchshift=0pt,% must be >= 0
         hatchcolor=black}
% declaring the pattern
\pgfdeclarepatternformonly[\hatchspread,\hatchthickness,\hatchshift,\hatchcolor]% variables
   {custom north west lines}% name
   {\pgfqpoint{\dimexpr-2\hatchthickness}{\dimexpr-2\hatchthickness}}% lower left corner
   {\pgfqpoint{\dimexpr\hatchspread+2\hatchthickness}{\dimexpr\hatchspread+2\hatchthickness}}% upper right corner
   {\pgfqpoint{\dimexpr\hatchspread}{\dimexpr\hatchspread}}% tile size
   {% shape description
    \pgfsetlinewidth{\hatchthickness}
    \pgfpathmoveto{\pgfqpoint{0pt}{\dimexpr\hatchspread+\hatchshift}}
    \pgfpathlineto{\pgfqpoint{\dimexpr\hatchspread+0.15pt+\hatchshift}{-0.15pt}}
    \ifdim \hatchshift > 0pt
      \pgfpathmoveto{\pgfqpoint{0pt}{\hatchshift}}
      \pgfpathlineto{\pgfqpoint{\dimexpr0.15pt+\hatchshift}{-0.15pt}}
    \fi
    \pgfsetstrokecolor{\hatchcolor}
%    \pgfsetdash{{1pt}{1pt}}{0pt}% dashing cannot work correctly in all situation this way
    \pgfusepath{stroke}
   }


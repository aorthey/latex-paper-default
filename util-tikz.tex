\pgfkeys{
/drawCapsule/.is family, /drawCapsule,
default/.style={line width = 1pt, color=black!30},
line width/.estore in = \linewidthC,
color/.estore in = \colorC,
}
\newcommand\drawCapsule[4][]{

        \pgfkeys{/drawCapsule, default, #1}
        \coordinate (A) at #2;
        \coordinate (B) at #3;
        \def\radius{#4}

        \coordinate (V1) at  ($(B)!\radius!-90:(A)$);
        \coordinate (V2) at  ($(A)!\radius!90:(B)$);
        \coordinate (V1P) at ($(B)!\radius!90:(A)$);
        \coordinate (V2P) at ($(A)!\radius!-90:(B)$);

        \draw[color=\colorC, line width = \linewidthC, #1] (A) circle (\radius);
        \draw[color=\colorC, line width = \linewidthC, #1] (B) circle (\radius);
        \draw[color=white,fill=white,line width=0pt] (V1) -- (V2) -- (V2P) -- (V1P) -- cycle;

        %\draw[color=white,fill=white,line width=2pt,shorten <= \linewidthCC, shorten >= \linewidthCC] (V1) -- (V1P);
        %\draw[color=white,fill=white,line width=2pt,shorten <= \linewidthCC, shorten >= \linewidthCC] (V2) -- (V2P);
        \draw[fill=\colorC,color=\colorC,shorten >= -\linewidthC, shorten <=
        -\linewidthC, line width = \linewidthC, #1] (V1) -- (V2);
        \draw[fill=\colorC,color=\colorC,shorten >= -\linewidthC, shorten <=
        -\linewidthC, line width = \linewidthC, #1] (V1P) -- (V2P);

        %%% draw middle points of each circle

        \draw[fill=\colorC,color=\colorC] (A) circle (1pt);
        \draw[fill=\colorC,color=\colorC] (B) circle (1pt);

}

\pgfdeclaredecoration{simple line}{initial}{
  \state{initial}[width=\pgfdecoratedpathlength-1sp]{\pgfmoveto{\pgfpointorigin}}
  \state{final}{\pgflineto{\pgfpointorigin}}
}

\tikzset{
   shift left/.style={decorate,decoration={simple line,raise=#1}},
   shift right/.style={decorate,decoration={simple line,raise=-1*#1}},
   node left/.style 2 args={
        decorate,decoration={raise=#1,markings,mark = at
        position 0.5 with {\node[circle, color=black,fill = white] {#2};}}
        },
}

%%% LeftShiftLine: Draw line between two nodes p1,p2, shift this line by D to the left and print a label in the middle of the line
%%% EXAMPLE: line between p1,p2 shifted by 0.5cm with the label $d$
%%% \LeftShiftLine[($(p1)$)][($(p2)$)][0.5cm][$d$];

\def\LeftShiftLine[#1][#2][#3][#4]{
    \begin{scope}[>=latex] % redef arrow for dimension lines
        \draw[dashed,|-|] #1 edge[shift left=#3] #2;
        \draw[] #1 edge[node left={#3}{#4}] #2;

    \end{scope}
}


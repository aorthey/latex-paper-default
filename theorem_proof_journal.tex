%% DEPENDENCIES (UBUNTU 12.04)
%% sudo apt-get install etoolbox

%% slightly modified version from:
%%http://tex.stackexchange.com/questions/33229/how-to-place-all-proofs-automatically-in-appendix

\usepackage{amsfonts} %mathbb, etc
\usepackage{amsthm} %%definitions
\usepackage{amsthm,amssymb}
\usepackage{etex,etoolbox} 
\usepackage{environ}
\makeatletter

\providecommand{\@fourthoffour}[4]{#4}
% We define an addition for the theorem-like environments; when
% \newtheorem{thm}{Theorem} is declared, the macro \thm expands
% to {...}{...}{...}{Theorem} and with \@fourthoffour we access
% to it; then we make available \@currentlabel (the theorem number)
% also outside the environment.  
\def\fixstatement#1{%
  \AtEndEnvironment{#1}{%
    \xdef\pat@label{\expandafter\expandafter\expandafter
      \@fourthoffour\csname#1\endcsname\space\@currentlabel}}}

% We allocate a block of 1000 token registers; in this way \prooftoks
% is 1000 and we can access the following registers of the block by
% \prooftoks+n (0<n<1000); we'll use a dedicated counter for it
% that is stepped at every proof
\globtoksblk\prooftoks{1000}
\newcounter{proofcount}

\NewEnviron{proofatend}{%
  \it{Proof in Appendix.}
  \edef\next{%
    \noexpand\begin{proof}[Proof of \pat@label]%
    \unexpanded\expandafter{\BODY}}%
  \global\toks\numexpr\prooftoks+\value{proofcount}\relax=\expandafter{\next\end{proof}}
  \stepcounter{proofcount}}

% \printproofs simply loops over the used token registers of the
% block, freeing their contents
\def\printproofs{%
  \count@=\z@
  \loop
    \the\toks\numexpr\prooftoks+\count@\relax
    \ifnum\count@<\value{proofcount}%
    \advance\count@\@ne
  \repeat}

\makeatother

\theoremstyle{definition}
\theoremstyle{plain}

\newtheorem{definition}{Definition}
\newtheorem{theorem}{Theorem}
\newtheorem{lemma}{Lemma}
\newtheorem{corollary}{Corollary}
\newtheorem{example}{Example}
\newtheorem{remark}{Remark}
\newtheorem{conjecture}{Conjecture}
\newtheorem{assumption}{Assumption}

\fixstatement{theorem}
\fixstatement{lemma}
\fixstatement{corollary}

%\renewcommand{\IEEEQED}{\IEEEQEDopen}
